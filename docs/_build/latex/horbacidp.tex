%% Generated by Sphinx.
\def\sphinxdocclass{report}
\documentclass[letterpaper,10pt,english]{sphinxmanual}
\ifdefined\pdfpxdimen
   \let\sphinxpxdimen\pdfpxdimen\else\newdimen\sphinxpxdimen
\fi \sphinxpxdimen=.75bp\relax

\PassOptionsToPackage{warn}{textcomp}
\usepackage[utf8]{inputenc}
\ifdefined\DeclareUnicodeCharacter
% support both utf8 and utf8x syntaxes
  \ifdefined\DeclareUnicodeCharacterAsOptional
    \def\sphinxDUC#1{\DeclareUnicodeCharacter{"#1}}
  \else
    \let\sphinxDUC\DeclareUnicodeCharacter
  \fi
  \sphinxDUC{00A0}{\nobreakspace}
  \sphinxDUC{2500}{\sphinxunichar{2500}}
  \sphinxDUC{2502}{\sphinxunichar{2502}}
  \sphinxDUC{2514}{\sphinxunichar{2514}}
  \sphinxDUC{251C}{\sphinxunichar{251C}}
  \sphinxDUC{2572}{\textbackslash}
\fi
\usepackage{cmap}
\usepackage[T1]{fontenc}
\usepackage{amsmath,amssymb,amstext}
\usepackage{babel}



\usepackage{times}
\expandafter\ifx\csname T@LGR\endcsname\relax
\else
% LGR was declared as font encoding
  \substitutefont{LGR}{\rmdefault}{cmr}
  \substitutefont{LGR}{\sfdefault}{cmss}
  \substitutefont{LGR}{\ttdefault}{cmtt}
\fi
\expandafter\ifx\csname T@X2\endcsname\relax
  \expandafter\ifx\csname T@T2A\endcsname\relax
  \else
  % T2A was declared as font encoding
    \substitutefont{T2A}{\rmdefault}{cmr}
    \substitutefont{T2A}{\sfdefault}{cmss}
    \substitutefont{T2A}{\ttdefault}{cmtt}
  \fi
\else
% X2 was declared as font encoding
  \substitutefont{X2}{\rmdefault}{cmr}
  \substitutefont{X2}{\sfdefault}{cmss}
  \substitutefont{X2}{\ttdefault}{cmtt}
\fi


\usepackage[Bjarne]{fncychap}
\usepackage{sphinx}

\fvset{fontsize=\small}
\usepackage{geometry}


% Include hyperref last.
\usepackage{hyperref}
% Fix anchor placement for figures with captions.
\usepackage{hypcap}% it must be loaded after hyperref.
% Set up styles of URL: it should be placed after hyperref.
\urlstyle{same}

\addto\captionsenglish{\renewcommand{\contentsname}{Contents:}}

\usepackage{sphinxmessages}
\setcounter{tocdepth}{1}



\title{horbac idp}
\date{Apr 22, 2021}
\release{0.1}
\author{Tsognong Fidèle}
\newcommand{\sphinxlogo}{\vbox{}}
\renewcommand{\releasename}{Release}
\makeindex
\begin{document}

\pagestyle{empty}
\sphinxmaketitle
\pagestyle{plain}
\sphinxtableofcontents
\pagestyle{normal}
\phantomsection\label{\detokenize{index::doc}}



\chapter{Indices and tables}
\label{\detokenize{index:indices-and-tables}}\begin{itemize}
\item {} 
\DUrole{xref,std,std-ref}{genindex}

\item {} 
\DUrole{xref,std,std-ref}{modindex}

\item {} 
\DUrole{xref,std,std-ref}{search}

\end{itemize}


\chapter{Introduction}
\label{\detokenize{index:introduction}}\begin{itemize}
\item {} 
This is a bulleted list.

\item {} 
It has two items, the second
item uses two lines.

\end{itemize}
\begin{enumerate}
\sphinxsetlistlabels{\arabic}{enumi}{enumii}{}{.}%
\item {} 
This is a numbered list.

\item {} 
It has two items too.

\item {} 
This is a numbered list.

\item {} 
It has two items too.

\end{enumerate}
\begin{itemize}
\item {} 
this is

\item {} 
a list
\begin{itemize}
\item {} 
with a nested list

\item {} 
and some subitems

\end{itemize}

\item {} 
and here the parent list continues

\end{itemize}
\begin{description}
\item[{term (up to a line of text)}] \leavevmode
Definition of the term, which must be indented

and can even consist of multiple paragraphs

\item[{next term}] \leavevmode
Description.

\end{description}

\begin{DUlineblock}{0em}
\item[] These lines are
\item[] broken exactly like in
\item[] the source file.
\end{DUlineblock}

This is a normal text paragraph again.


\begin{savenotes}\sphinxattablestart
\centering
\begin{tabulary}{\linewidth}[t]{|T|T|T|T|}
\hline
\sphinxstyletheadfamily 
Header row, column 1
(header rows optional)
&\sphinxstyletheadfamily 
Header 2
&\sphinxstyletheadfamily 
Header 3
&\sphinxstyletheadfamily 
Header 4
\\
\hline
body row 1, column 1
&
column 2
&
column 3
&
column 4
\\
\hline
body row 2
&
…
&
…
&\\
\hline
\end{tabulary}
\par
\sphinxattableend\end{savenotes}

This is a paragraph that contains \sphinxhref{https://domain.invalid/}{a link}.


\chapter{This is a heading}
\label{\detokenize{index:this-is-a-heading}}\begin{quote}\begin{description}
\item[{fieldname}] \leavevmode
Field content

\end{description}\end{quote}

\noindent\sphinxincludegraphics{{bank}.png}

Lorem ipsum %
\begin{footnote}[1]\sphinxAtStartFootnote
Text of the first footnote.
%
\end{footnote} dolor sit amet … %
\begin{footnote}[2]\sphinxAtStartFootnote
Text of the second footnote.
%
\end{footnote}

Lorem ipsum \sphinxcite{index:ref} dolor sit amet.

This is a normal text paragraph. The next paragraph is a code sample:

\begin{sphinxVerbatim}[commandchars=\\\{\}]
\PYG{n}{It} \PYG{o+ow}{is} \PYG{o+ow}{not} \PYG{n}{processed} \PYG{o+ow}{in} \PYG{n+nb}{any} \PYG{n}{way}\PYG{p}{,} \PYG{k}{except}
\PYG{n}{that} \PYG{n}{the} \PYG{n}{indentation} \PYG{o+ow}{is} \PYG{n}{removed}\PYG{o}{.}

\PYG{n}{It} \PYG{n}{can} \PYG{n}{span} \PYG{n}{multiple} \PYG{n}{lines}\PYG{o}{.}
\end{sphinxVerbatim}

This is a normal text paragraph again.

\begin{sphinxthebibliography}{Ref}
\bibitem[Ref]{index:ref}
Book or article reference, URL or whatever.
\end{sphinxthebibliography}



\renewcommand{\indexname}{Index}
\printindex
\end{document}